\documentclass[a4paper,11pt]{article}

\usepackage[utf8]{inputenc}
\usepackage[T1]{fontenc}
\usepackage[margin=1in]{geometry}
\usepackage{graphicx}
\usepackage{xcolor}
\usepackage{enumitem}
\usepackage{fancyhdr}
\usepackage{setspace}
\usepackage{booktabs}

% 自定义颜色
\definecolor{highlight}{RGB}{0,82,147}
\definecolor{sectioncolor}{RGB}{41,128,185}

% 页眉页脚设置
\pagestyle{fancy}
\fancyhf{}
\rhead{\footnotesize Policy Brief}
\lhead{\footnotesize Your Organization}
\rfoot{\thepage}

% 段落和行距
\onehalfspacing
\setlength{\parindent}{0pt}
\setlength{\parskip}{0.8em}

% 自定义命令
\newcommand{\keyword}[1]{\textbf{#1}}
\newcommand{\boxedtext}[1]{\noindent\colorbox{highlight!10}{\parbox{\dimexpr\linewidth-2\fboxsep}{#1}}}

\title{\color{highlight}\textbf{The economic case for climate finance at scale}}
\author{Kaibiao ZHU, Hongyue WU, Rongjia XU}
\date{}

\begin{document}

\maketitle

% 执行摘要
\section*{Executive Summary}
\boxedtext{
\textbf{IT WILL BE impossible to contain the global temperature rise to 1.5 to 2 degrees Celsius above pre-industrial levels unless emerging market and developing economies (EMDEs) decarbonise much more rapidly. This policy brief examines the economic case for advanced-country financial support for replacement of coal with renewable energy sources in EMDEs.}
}

\vspace{1em}
\hrule
\vspace{1.5em}

% 正文开始
\section{Introduction}
Global carbon emissions are at a historic high. Emissions in 2023 consumed 10.67 percent of the remaining carbon budget consistent with limiting global warming to 1.5 degrees Celsius compared to pre-industrial levels.

\section{The desirability of climate finance at scale}
We use a dataset of estimates of the costs and benefits of phasing out coal use – the largest single source of carbon emissions – and replacing the phased-out coal energy with renewable energy.

\subsection{The global net benefits to decarbonisation are very large}
Table 1 sets out the global costs and benefits from a gradual coal phase-out, consistent with achieving net-zero emissions by 2050.

\begin{table}[h]
\centering
\caption{The global costs and benefits of Paris Agreement-consistent coal phase-out}
\begin{tabular}{lccc}
\toprule
& 2024-2030 & 2024-2050 & 2024-2100 \\
\midrule
\textbf{Costs (in \$ trillions)} & 12.2 & 23.5 & 36.0 \\
\quad Advanced countries & 3.8 & 6.9 & 10.5 \\
\quad EMDEs & 8.5 & 16.6 & 25.4 \\
\bottomrule
\end{tabular}
\end{table}

\section{Conclusion}
Our results imply that there is a strong economic case for wealthy countries to provide climate finance at scale, beyond their moral obligations under the Paris Agreement.

\end{document}