\documentclass[a4paper,11pt]{article}


\usepackage[T1]{fontenc}
\usepackage{lmodern}
\usepackage[margin=1in]{geometry}
\usepackage{graphicx}
\usepackage{xcolor}
\usepackage{enumitem}
\usepackage{fancyhdr}
\usepackage{setspace}
\usepackage{booktabs}
\usepackage{times}


% 颜色
\definecolor{highlight}{RGB}{0,82,147}
\definecolor{sectioncolor}{RGB}{41,128,185}

% 页眉页脚
\pagestyle{fancy}
\fancyhf{}
\rhead{\footnotesize Policy Brief}
\lhead{\footnotesize HKUST(GZ) Foundation in Green and Sustainable Finance}
%\lhead{\footnotesize HKUST(GZ)}
\rfoot{\thepage}


\onehalfspacing
\setlength{\parindent}{0pt}
\setlength{\parskip}{0.8em}


\newcommand{\keyword}[1]{\textbf{#1}}
\newcommand{\boxedtext}[1]{\noindent\colorbox{highlight!10}{\parbox{\dimexpr\linewidth-2\fboxsep}{#1}}}

\title{\color{highlight}\textbf{China's Sustainable Aviation Fuel: How to Break Through the Bottleneck of Large-scale Production?}}
\author{Kaibiao ZHU, Hongyue WU, Rongjia XU}
\date{}

\begin{document}

\maketitle


\section*{Executive Summary}
\boxedtext{
\textbf{At present, the scale-up of China's SAF (Sustainable Aviation Fuel) faces three core bottlenecks: Firstly, there are only directional goals but no binding requirements for blending or carbon intensity, resulting in unstable market expectations; Secondly, the raw materials mainly consist of waste oils, with scattered collection and transportation and significant cross-border mismatch; Thirdly, the cost is significantly higher than that of fossil jet fuel, and the incentives are weak. The existing accounting rules have not yet been integrated with the price mechanism. The corresponding solutions are as follows: Firstly, through the formulation of laws, implement mandatory blending in stages and provide supporting mechanisms such as tradable certificates and compliance cost caps to create a predictable and rigid demand; Secondly, establish a "Southeast Asia - China" raw material coordination mechanism and a unified book-and-claim platform across the country to alleviate supply mismatch and traceability problems; Finally, introduce production-side support based on the lifecycle carbon intensity, directly offsetting price differences and closing the loop with domestic accounting rules.
}
}

\vspace{1em}
\hrule
\vspace{1.5em}


\section{Introduction}
Aviation is one of the "difficult-to-decarbonize" industries, and in the near and medium term, it still heavily relies on liquid fuels. Sustainable Aviation Fuel (SAF) can be seamlessly integrated with existing aircraft and fuel supply systems and is widely regarded as the most feasible main path for the 2030s. Internationally, several economies have adopted "pre-defined rules, then attract investment" approaches, such as mandatory blending, certificate trading, and tax incentives. Domestically, multiple rounds of demonstrations have been completed, application pilot projects have been initiated, and standardization construction has been advanced. However, there are still constraints in terms of cost, raw materials, and market mechanisms. This policy brief answers: why has SAF not achieved large-scale production after multiple rounds of demonstrations in China, and how each key bottleneck should be quickly resolved. This article focuses on policies and market mechanisms, targeting decision-makers and industry entities who need to allocate resources and formulate rules based on this. It mainly covers domestic policies and research over the past five years, and explores three key bottlenecks: the lack of binding demand signals, the mismatch between raw materials and supply, and the incomplete price mechanism and accounting. And for each issue, an actionable option is proposed, and compared with international practices to verify feasibility and implementation paths.

\section{Overview of the problems}
1

\subsection{The global net benefits to decarbonisation are very large}
Table 1 sets out the global costs and benefits from a gradual coal phase-out, consistent with achieving net-zero emissions by 2050.

\begin{table}[h]
\centering
\caption{The global costs and benefits of Paris Agreement-consistent coal phase-out}
\begin{tabular}{lccc}
\toprule
& 2024-2030 & 2024-2050 & 2024-2100 \\
\midrule
\textbf{Costs (in \$ trillions)} & 12.2 & 23.5 & 36.0 \\
\quad Advanced countries & 3.8 & 6.9 & 10.5 \\
\quad abc & 8.5 & 16.6 & 25.4 \\
\bottomrule
\end{tabular}
\end{table}

\section{Examination of the findings and options}
Our results imply that there is a strong economic case for wealthy countries to provide climate finance at scale, beyond their moral obligations under the Paris Agreement.


\section{References}


\end{document}