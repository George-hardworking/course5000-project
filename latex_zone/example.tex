\documentclass[a4paper,11pt]{article}
\usepackage[utf8]{inputenc}
\usepackage[T1]{fontenc}
\usepackage[margin=1in]{geometry}
\usepackage{graphicx}
\usepackage{wrapfig}
\usepackage{xcolor}
\usepackage{enumitem}
\usepackage{fancyhdr}
\usepackage{setspace}
\usepackage{lipsum} % 仅用于占位,实际使用时删除

% 自定义颜色(用于强调)
\definecolor{highlight}{RGB}{0,82,147} % 深蓝色,专业感

% 页眉页脚
\pagestyle{fancy}
\fancyhf{}
\rhead{\footnotesize Policy Brief}
\lhead{\footnotesize Your Organization Name}
\rfoot{\thepage}

% 段落间距
\onehalfspacing

% 自定义命令:关键信息框(用于侧边栏)
\newcommand{\keypoint}[1]{%
  \noindent\colorbox{highlight!10}{\parbox{\dimexpr\linewidth-2\fboxsep}{\textbf{Key Insight:} #1}}\vspace{0.5em}
}

\title{\color{highlight}\textbf{We Think in Advance for the Benefit of the Environment}}
\author{}
\date{}

\begin{document}

\maketitle

% 摘要(Executive Summary)
\noindent\textbf{\large Executive Summary}\\[0.5em]
This policy brief highlights the urgent need to strengthen environmental impact assessments (EIAs) in national planning. Current practices fail to integrate climate risks, leading to costly infrastructure failures. We recommend mandating climate-resilient EIA protocols by 2026, with phased implementation led by the Ministry of Environment.

\vspace{1em}
\hrule
\vspace{1.5em}

\section*{Why Change Is Needed}
Environmental impact assessments (EIAs) remain a procedural formality in many regions. Despite legal requirements, over 60\% of major projects approved in 2023 omitted climate vulnerability analyses. This gap has resulted in repeated flood damage to transport infrastructure, costing taxpayers \$2.1 billion last year alone.

\keypoint{60\% of EIAs ignore climate risks — a systemic oversight with real financial consequences.}

\section*{Key Research Findings}
Our analysis of 120 EIA reports (2020–2024) reveals:
\begin{itemize}[left=0pt]
    \item Only 18\% included future climate scenarios.
    \item Projects without climate screening were 3.2× more likely to suffer damage within 5 years.
    \item Public consultation was tokenistic in 74\% of cases.
\end{itemize}

\begin{figure}[h]
\centering
\includegraphics[width=0.6\textwidth]{example-image} % 使用 example-image 作为占位符
\caption{Correlation between EIA quality and project resilience (2020–2024)}
\end{figure}

\section*{Policy Options}
Three alternatives were evaluated:
\begin{enumerate}
    \item \textbf{Status quo}: Maintain current EIA framework — low cost, high risk.
    \item \textbf{Voluntary guidelines}: Encourage but don't enforce climate screening — moderate effectiveness.
    \item \textbf{Mandatory climate-resilient EIA}: Legally require integrated climate risk analysis — high upfront cost, long-term savings.
\end{enumerate}

We advocate for Option 3, as it aligns with international best practices (e.g., EU Directive 2023/XX) and offers the highest cost-benefit ratio.

\section*{Recommendations}
\textbf{Short-term (2025–2026):}
\begin{itemize}
    \item \textbf{Ministry of Environment}: Draft revised EIA regulations by Q2 2025.
    \item \textbf{Local governments}: Pilot climate-EIA in 5 high-risk provinces.
\end{itemize}

\textbf{Long-term (2027+):}
\begin{itemize}
    \item Integrate EIA data into national climate adaptation dashboard.
    \item Require independent review of all EIAs for projects over \$10M.
\end{itemize}

\vspace{2em}
\noindent\rule{\textwidth}{0.4pt}\\
\footnotesize
\textbf{Contact:} policy@yourorg.org \quad | \quad www.yourorg.org/policy-briefs \\
\textbf{Based on:} "Implementing Environmental Impact Assessment in the Context of European Integration" (Working Paper No. 12, 2024)

\end{document}